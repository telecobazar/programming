\documentclass[10pt,a4paper]{article}
\usepackage[utf8]{inputenc}
\usepackage[spanish]{babel}
\usepackage{amsmath}
\usepackage{amsfonts}
\usepackage{amssymb}
\selectlanguage{spanish}
\title{EN PROGRAMACIÓN - INICIOS}
\author{JESÚS MARÍA CALDERÓN - GITHUB JELUCHU}
\begin{document}
\maketitle
\center Apuntes programados con \LaTeX{}
\bigskip
\begin{enumerate}
\item \textbf{Tipos de datos} \\
Los tipos de datos permitidos para pseudocódigo son:
\begin{enumerate}
\item \textbf{ENTERO:} Representa valores enteros positivos y negativos. Se representa con 'int'.
\item \textbf{REAL:} Representa valores reales con 35 dígitos de precisión. Se representa con 'float' o 'double' (para mayor precisión).
\item \textbf{CARÁCTER:} Representa valores alfanuméricos. Se representa con 'char'.
\item \textbf{LÓGICO:} Representa valores lógicos. Se representa con 'bool'. \\ (VERDADERO O FALSO)
\end{enumerate}
\medskip
\item \textbf{Variables y Constantes} \\
Un dato es toda aquella información relevante que puede ser tratada con posterioridad en un programa. Según el modo de almacenamiento, existen dos tipos de datos: variables y constantes
\begin{enumerate}
\item \textbf{Variables:} Almacenan temporalmente datos en la memoria y podemos cambiar su contenido durante el programa. Los fundamentales son: \textbf{void, char, int, float y double}, en los nuevos estándares se incluye también el tipo \textbf{bool y enum}. Modificadores de variables: permiten ajustar ciertas propiedades de cada tipo: \textbf{short y long}.
\smallskip
\item \textbf{Constantes:} Indica que el valor de la variable no se va a modificar en el transcurso del algoritmo.
\smallskip \\
\center \textbf{Su sintaxis es: \\
const  $<tipo dato>   <nombre constante>   =   <valor>$;} 
\medskip \center \textbf{Ejemplo:   const float cambio = 0.92}  
\end{enumerate}
\end{enumerate}

\end{document}